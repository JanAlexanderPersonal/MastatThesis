

\subsection{UniSiegen dataset\label{sec:DataUSiegen}}

This dataset is made available in 2014 by the University of Siegen, Germany.
Dr D. Zukic \cite{Zukic2014} constructed it as part of his PhD project.
This dataset contains 26 \acrshort{mri} scans of 17 different patients\footnote{This is not clearly stated, but can be inferred from the metadata.}. 

The fact that scans of the same patient are correlated will be taken into account in the train, validation and test split.
For more details on this split, see section \ref{sec:trainValTestSplit} on page \pageref{sec:trainValTestSplit}.


\subsubsection{Original Objective of the Dataset}

This dataset was collected from several hospitals (Sarajevo, Marburg, Brisbane, Schwabach, Bad Wildungen \& Prague). The MRI scanner settings were varied between the scans (T1, T2, TIRM).
The PhD project objective was to build a segmentation model to automate the segmentation of the lumbar vertebrae in the \acrshort{mri} scans to facilitate the diagnosis of several spine pathologies 
such as scoliosis, spondylolisthesis \footnote{Spondylolisthesis is the displacement of one spinal vertebra compared to another.} and vertebral fractures.
The final model developed by dr. D. Zukic consisted of a Viola-Jones detector for detection and vertebral body size approximation.
The average Dice score compared to the manual reference was reported to be 79.3\%.

\subsubsection{Patient statistics}
\marginpar{
        \includegraphics[width=5cm]{automated_graphs/USiegen_ageboxplot.png}
        \captionof{figure}{USiegen patients age distribution}
        \label{fig:USiegen_Age}
    }
In \cite{Zukic2014}, it is not entirely made clear which scans are taken from the same patient.
It is made clear, however, that the 26 scans were not obtained from 26 patients.
The information was inferred from the naming of the scans and the provided gender en age information\footnote{
    Wrongfully assuming two scans come from the same patient does not cause data leakage.
}.

Figure \ref{fig:USiegen_Age} illustrates that the USiegen dataset contains almost double the number of female patients compared to male patients.
These patients are relatively young compared to the patients in the \textit{xVertSeg} dataset.

Only three of the patients in this dataset were categorized as having no spinal pathologies.

\subsubsection{Technical information}

Several \acrlong{mri} techniques were used to obtain the dataset: T1, T2 \& TIRM.
This factor is not taken into account in the model development or the dataset split.

The volumes in the USiegen dataset are strongly cropped. 
Both in the anteroposterior and the craniocaudal direction, the volumes are on average 370 mm.
In the left-right direction, however, the volumes have been cropped severely. The volumes are, on average, only 68 mm wide.
The images have been cropped to only include the \acrshort{roi}.
Along this left-right dimension, the voxel spacing is large. 

\begin{SCfigure}[][htb]
    \centering
    \includegraphics[width=.95\textwidth]{automated_graphs/USiegen_Aka3.pdf}
    \caption{USiegen dataset scan \textit{Aka3}. \label{fig:USiegen_Aka3}. It is immediately clear the USiegen volumes are cropped differently than the xVertSeg volumes.
    In the craniocaudal direction, both sacrum and coccyx are visible. Along the left-righ axis, the volumes have been cropped severely.}
\end{SCfigure}
\begin{SCfigure}[][htb]
    \centering
    \includegraphics[width=.95\textwidth]{images/USiegen004_s20_mask.pdf}
    \caption{\label{fig:USiegen004_s20_mask}Sagittal slice of USiegen dataset scan compared with the \Gls{groundtruth} mask for this scan.
    This image illustrates a spine with a crushed vertebra. 
    It can also be remarked that the ground truth masks provided for the USiegen dataset only contain the vertebra body, not the vertebra laminae.
    \protect\input{tables/colourlegend.tex}
    }
\end{SCfigure}

The original scans in the \textit{USiegen} dataset were cropped in the \textit{left-right} direction. 
Although the scan resolution is relatively high in the Sagittal planes, the slice spacing along the left-right axis is coarser (see figure \ref{fig:AllDataset_dims}).  
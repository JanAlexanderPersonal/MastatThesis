\section{Train, validation and test split considerations\label{sec:trainValTestSplit}}

To allow evaluation of the models produced, a test set is split from the development set.
The objective is to evaluate how well a model generalizes to new data\footnote{
    New data is to be interpreted here as a new sample from \textit{the same population} as the development sample.
    Unfortunately, it cannot be claimed that the different datasets I collected indeed form a perfect representation of the population of medical images of the lumbar spine.}. 
To provide an honest estimate of the out-of-sample performance, the test dataset should represent the investigated population and (hidden) correlations with elements of the development sample should be avoided.

This is done\footnote{
    To accomplish this, I used the function \texttt{GroupedStratifiedSplit} at \url{https://github.com/scikit-learn/scikit-learn/pull/18649/}. 
    This class is not yet part of the official \texttt{sci-kit learn} library release (at the time of writing) but functions well for this application.} taking into account the following:
\begin{description}
    \item[Stratified data:] The combination of data from different sources is stratified. Every source is considered a subpopulation. The data split is made such that the proportion of scans originating from each data source in every split is proportional to their occurrence in the total population.
    \item[Grouped data:] The scans of the same patient can be assumed to be correlated to each other. These scans should not be spread over different splits. The data is split at patient level.
\end{description}

For each datasource, the intended split is $\frac{4}{6}$ for train set, $\frac{1}{6}$ for cross validation set and $\frac{1}{6}$ for test set.
This distribution is not perfect but acceptable, as is indicated by the values in table \ref{tab:summary_split}.

\begin{SCtable}[\sidecaptionrelwidth][h]
 
    \begin{tabular}{lrrr}
\toprule
split &  test &  train &  xval \\
source       &       &        &       \\
\midrule
MyoSegmenTUM &     9 &     33 &     9 \\
PLoS         &     4 &     14 &     4 \\
USiegen      &     2 &     14 &     1 \\
xVertSeg     &     2 &     10 &     3 \\
\bottomrule
\end{tabular}

    \caption{Number of volumes by datasource and by split. The scan volumes are split such that scans of the same patient are in the same split set.\label{tab:summary_split}}
  
  \end{SCtable}

  \begin{SCtable}[\sidecaptionrelwidth][h]
 
    \begin{tabular}{ll|lll|l}
    \toprule
    split        &            & test & train & xval & total \\
    source       & dimension  &      &       &      &       \\ \midrule
    MyoSegmenTUM & Transverse & 1989 & 7293  & 1989 & 11271 \\
                 & Frontal    & 1989 & 7293  & 1989 & 11271 \\
                 & Sagittal   & 720  & 2650  & 725  & 4095  \\
    PLoS         & Transverse & 1528 & 5348  & 1528 & 8404  \\
    USiegen      & Transverse & 733  & 5063  & 501  & 6297  \\
                 & Frontal    & 681  & 4276  & 1080 & 6037  \\
                 & Sagittal   & 145  & 1089  & 180  & 1414  \\
    xVertSeg     & Transverse & 759  & 2652  & 785  & 4196  \\
                 & Frontal    & 812  & 4163  & 1290 & 6265  \\
                 & Sagittal   & 812  & 4163  & 1290 & 6265  \\ \midrule
    total        & Transvers  & 5009 & 20356 & 4803 & 30168 \\
                 & Frontal    & 3482 & 15732 & 4359 & 23573 \\
                 & Sagittal   & 1677 & 7902  & 2195 & 11774 \\ \bottomrule
    \end{tabular}
    \caption{Number of slices by data source and by split.
    These values might seem high, yet, the reader should not forget that these image slices are highly correlated. 
    There is very little additional independent information comparing one slice with a slice taken just 1mm further.
    On top of this, many slices only contain the background class without providing the model information to train on the lumbar vertebrae classes. 
    \label{tab:summary_split_slices}}
  
  \end{SCtable}

The result of this operation is detailed in appendix \ref{sec:appendix_split} on page \pageref{sec:appendix_split}.
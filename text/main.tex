\documentclass[
    a4paper, % Stock and paper size.
    9pt, % Type size.
    % article,
    oneside, 
    onecolumn, % Only one column of text on a page.
    % openright, % Each chapter will start on a recto page.
    % openleft, % Each chapter will start on a verso page.
    openany, % A chapter may start on either a recto or verso page.
    ]{memoir}
\usepackage[utf8]{inputenc}
\usepackage[T1]{fontenc}
\usepackage{lmodern}
\usepackage[final]{microtype}
\usepackage[dvips]{graphicx}
\usepackage{xcolor}
\usepackage{tikz}
\usepackage{caption}
%\usepackage{times}

% Comment in final version
\usepackage{todonotes}
\usepackage{soul}

%\usepackage[margincaption,outercaption,ragged,wide]{sidecap}
\usepackage[margincaption,ragged]{sidecap}
\sidecaptionvpos{figure}{t} 
\sidecaptionvpos{table}{t}

\usepackage[
breaklinks=true,colorlinks=true,
%linkcolor=blue,urlcolor=blue,citecolor=blue,% PDF VIEW
linkcolor=black,urlcolor=black,citecolor=black,% PRINT
bookmarks=true,bookmarksopenlevel=2]{hyperref}

\usepackage{geometry}
% PDF VIEW
% \geometry{total={210mm,297mm},
% left=25mm,right=25mm,%
% bindingoffset=0mm, top=25mm,bottom=25mm}
% PRINT
% \geometry{total={210mm,297mm},left=20mm,right=50mm,bindingoffset=10mm, top=25mm,bottom=25mm}

% Code from https://tex.stackexchange.com/questions/275565/tufte-layout-in-painless-memoir

% Start code

% MEMOIR LAYOUT
\setlength{\baselineskip}{14pt}
\setlength{\normalbaselineskip}{14pt}

% GEOMETRY
\settrims{0pt}{0pt}
\settypeblocksize{49\baselineskip}{107mm}{*}
\setlrmargins{24.8mm}{*}{*}
\setulmargins{27.4mm}{*}{*}
\setheadfoot{\baselineskip}{\baselineskip}
\setheaderspaces{*}{2\baselineskip}{*}
\setmarginnotes{8.2mm}{49.4mm}{\onelineskip}
\checkandfixthelayout


% TEXT
% ragged2e provides ragged justification with hyphenation
\RequirePackage{ragged2e}
\AtBeginDocument{\RaggedRight}
\setmpjustification{\RaggedLeft}{\RaggedRight}
\setlength{\RaggedRightParindent}{1.0pc}
\setlength{\parindent}{1pc}
\setlength{\parskip}{0pt}
% linespacing ~ 14pt
\linespread{1.17}

% text styling of all side footnotes
\renewcommand{\footnotesize}{\fontsize{8pt}{10pt}\selectfont}
\renewcommand{\foottextfont}{\footnotesize}
% styling and placement of mark
\footmarkstyle{{#1. }}
\setlength{\footmarkwidth}{0em}
\setlength{\footmarksep}{-\footmarkwidth}
% memoir command - do all footnotes in margin
\footnotesinmargin


% SIDECAPTIONS
\setsidecaps{\marginparsep}{\marginparwidth}
\sidecapmargin{outer}
\setsidecappos{t}
\renewcommand*{\sidecapstyle}{%
\captionnamefont{\foottextfont\scshape}
\ifscapmargleft
\captionstyle{\RaggedLeft\footnotesize\foottextfont}%
\else
\captionstyle{\RaggedRight\footnotesize\foottextfont}%
\fi}

% FULLWIDTH environment
% The following code should be used *after* any changes to the margins and
% page layout are made (e.g., after the geometry package has been loaded).
\newlength{\fullwidthlen}
\setlength{\fullwidthlen}{\marginparwidth}
\addtolength{\fullwidthlen}{\marginparsep}

% \newlength{\fullwidthlen}
% \setlength{\fullwidthlen}{\marginparwidth}
% \addtolength{\fullwidthlen}{\marginparsep}

\newenvironment{fullwidth}{%
  \begin{adjustwidth*}{}{-\fullwidthlen}%
}{%
  \end{adjustwidth*}%
}


% End code

%

\OnehalfSpacing
%\linespread{1.3}

% disable by disabling the todo notes
\makeatletter
 \if@todonotes@disabled
 \newcommand{\hlnote}[2]{#1}
 \else
 \newcommand{\hlnote}[2]{\todo{#2}\texthl{#1}}
 \fi
 \makeatother

%%% CHAPTER'S STYLE
%\chapterstyle{bianchi}
%\chapterstyle{ger}
\chapterstyle{dash}
%\chapterstyle{madsen}
%\chapterstyle{ell}
%%% STYLE OF SECTIONS, SUBSECTIONS, AND SUBSUBSECTIONS
\setsecheadstyle{\Large\bfseries\sffamily\raggedright}
\setsubsecheadstyle{\large\bfseries\sffamily\raggedright}
\setsubsubsecheadstyle{\bfseries\sffamily\raggedright}


%%% STYLE OF PAGES NUMBERING
\pagestyle{companion}\nouppercaseheads 
\pagestyle{headings}
\pagestyle{Ruled}
%\pagestyle{plain}
\makepagestyle{plain}
\makeevenfoot{plain}{\thepage}{}{}
\makeoddfoot{plain}{}{}{\thepage}
\makeevenhead{plain}{}{}{}
\makeoddhead{plain}{}{}{}

%%% For parts with abstracts on the page

\usepackage{xpatch}

\makeatletter
\xpatchcmd{\part}{\null\vfil}{\vspace*{.1\textheight}}{}{}

\providecommand{\abstractname}{Content of this part}
\newenvironment{partwithabstract}
  {\begingroup\let\@endpart\relax\part@withabstract}
  {\endquotation\endgroup\@endpart}
\newcommand{\part@withabstract}{\@dblarg\part@@withabstract}
\def\part@@withabstract[#1]#2{%
  \part[#1]{#2}%
  \vfil
  \begin{center}\bfseries\abstractname\vspace{-.5em}\vspace{\z@}\end{center}
  \quotation
}
\makeatother




\maxsecnumdepth{subsection} % chapters, sections, and subsections are numbered
\maxtocdepth{subsection} % chapters, sections, and subsections are in the Table of Contents

\usepackage[backend=biber]{biblatex}
\addbibresource{library.bib}

%% make a glossary
\usepackage[acronym, xindy, nonumberlist]{glossaries}
\makeglossaries


\newglossaryentry{weaklysupervisedl}
{
        name={Weakly Supervised},
        description={Weakly supervised machine learning where the ground truth labels are only partially available. 
        In the context of image segmentation, this can mean that the labels are only provided at image level or that point level annotation in the image is provided.
        The model is trained on incomplete labels, the desired result remains the complete segmentation of the image.
        Just like in the case of \textit{Fully Supervised Learning} the objective is to model the relationship between an \textit{input} and an \textit{output}. 
        Due to the labels being incomplete, there is a stronger need to identify the internal structure of the data, as is the case for \textit{Unsupervised learning}.
        }
}

\newglossaryentry{segmentation}{
        name={Segmentation},
        description={
                The \textit{segmentation} problem in machine vision consists of the classification of each individual pixel or voxel. 
                The problem of \textit{semantic} segmentation is to detect, for each pixel, the object category it belongs to. 
                \textit{Instance} segmentation digs deeper. It identifies for each pixel the object instance it belongs to.
                The difference is that it differentiates between two objects of the same object category in the picture. 
                }
        }

\newglossaryentry{groundtruth}{
        name={Ground Truth},
        description={
                The \textit{Ground Truth} is a term used in machine learning to indicate the ideal expected result. 
                In the context of Instance Segmentation, the ground truth is the true class of every pixel or voxel. 
        }
        }

\newglossaryentry{unsupervisedl}
{
        name={Unsupervised},
        description={
                In an \textit{Unsupervised} machine learning problem, no labels are present. 
                The aim is not to model the relationship between an \textit{input} and an \textit{output}, the aim is to model the structure of the data.
                Frequent applications of these techniques are clustering and dimensionality reduction of data. 
                }
}

\newglossaryentry{supervisedl}
{
        name={Supervised},
        description={
                (Fully) Supervised Machine Learning task where target labels are present. 
                The objective of these problems is to model the relationship between an \textit{input} and an \textit{output}.
                }
}

\newglossaryentry{tomography}
{
        name={Tomography},
        description={
                Imaging of a volume through the use of a penetrating wave. 
                Through these waves, a collection of images, called \textit{tomograms}, are produced.
                The mathematical procedure to reconstruct the original volume based on these images is called \textit{tomographic reconstruction}.
                A \acrfull{ct} scan is produced through tomographic reconstruction of several X-ray radiographs.
                }
}

\newglossaryentry{machinevision}
{
        name={Machine vision},
        description={
                The branch of Artificial Intelligence with the objective of invering results from images. 
                In this work, these images can be both two dimensional (\textit{pictures}) as three dimensional (\textit{volumes}).
                }
}

\newglossaryentry{ai}
{
        name={Artificial Intelligence},
        description={
                The study of using computers to automatically perform tasks which once were considered only humans could do.
                This includes, but is not restricted to, interpretation of speech and images. It is often refered to with the acronym \textsc{ai}.
                }
}

\newglossaryentry{deepl}
{
        name={Deep Learning},
        description={
                Deep learning is a branch of Machine Learning where a set of multiple sequential layers is used to progressively extract higher-level features from the raw input data.
                }
}

\newglossaryentry{caml}
{
        name={Class Activation Map},
        description={
                Technique to identify region of an image \textit{responsible} for the classification result.
        },
        first={Class Activation Map (CAM)},
        text={CAM}
}

\newglossaryentry{lc-fcn}{
        name={LC-FCN},
        description={
                \acrfull{fcn} with a \acrfull{lc} as optimization loss.
                This technique was introduced by I. Laradji et al in \cite{Laradji2018} for the training of an automated instance counting network on point-supervised data.
                Later it is applied in \cite{Laradji2020} as the Location branch of the WISE network.
                },
        first={Fully Convolutional Network with Location based Counting loss (LC-FCN)},
        text={LC-FCN}
}

\newglossaryentry{wise}{
        name={WISE},
        description={Weakly-supervised Instance SEgmentation model (\cite{Laradji2020})},
        first={Weakly-supervised Instance SEgmentation model (WISE)},
        text={WISE}
        }

\newglossaryentry{features}{
        name={Features},
        description={Transformation of the input image.},
        text={features}
        }


\newacronym{cnn}{CNN}{Convolutional Neural Network}
\newacronym{crf}{CRF}{Conditional Random Field}
\newacronym{ct}{CT}{Computer Tomography}
\newacronym{mri}{MRI}{Magnetic Resonance Imaging}
\newacronym{ml}{ML}{Machine Learning}
\newacronym{mil}{MIL}{Multi-Instance Learning}
\newacronym{rnn}{RNN}{Recurrent Neural Network}
\newacronym{us}{US}{Ultra Sound imaging}
\newacronym{fcn}{FCN}{Fully Convolutional Neural Network}
\newacronym{lc}{LC}{Location based Counting loss}
\newacronym{iou}{IoU}{Intersection over Union}
\newacronym{roi}{RoI}{Region of Interest}
\newacronym{osf}{OSF}{Open Science Foundation}

%%%---%%%---%%%---%%%---%%%---%%%---%%%---%%%---%%%---%%%---%%%---%%%---%%%

\begin{document}

\newgeometry{total={210mm,297mm},left=30mm,right=30mm,bindingoffset=5mm, top=25mm,bottom=25mm} 

%%%---%%%---%%%---%%%---%%%---%%%---%%%---%%%---%%%---%%%---%%%---%%%---%%%
%   TITLEPAGE
%
%   due to variety of titlepage schemes it is probably better to make titlepage manually
%
%%%---%%%---%%%---%%%---%%%---%%%---%%%---%%%---%%%---%%%---%%%---%%%---%%%
\thispagestyle{empty}

{%%%
\sffamily
\centering
\Large

~\vspace{\fill}

{\huge 
Automated segmentation of the human spine on CT images from point-level labels.
}

\vspace{2.5cm}

{\LARGE
Jan Alexander
}

\vspace{3.5cm}

A thesis submitted in partial fulfillment for the\\
degree of Master in Statistical Data Analysis\\[1em]
in the\\[1em]
Faculty of Science\\
Universiteit Gent

\vspace{3.5cm}

Supervisors:\\
Dr. Joris Roels
Dr. Bert Vankeirsbilck

\vspace{\fill}

September 2021

%%%
}%%%

\cleardoublepage
%%%---%%%---%%%---%%%---%%%---%%%---%%%---%%%---%%%---%%%---%%%---%%%---%%%
%%%---%%%---%%%---%%%---%%%---%%%---%%%---%%%---%%%---%%%---%%%---%%%---%%%

\tableofcontents*

\clearpage

%%%---%%%---%%%---%%%---%%%---%%%---%%%---%%%---%%%---%%%---%%%---%%%---%%%
%%%---%%%---%%%---%%%---%%%---%%%---%%%---%%%---%%%---%%%---%%%---%%%---%%%

\glsaddall
\setglossarystyle{listdotted}
\printglossary[type=\acronymtype]
\glossarystyle{altlist}
\printglossary

\clearpage

%\cleardoublepage
\restoregeometry
\chapter*{Abstract}
\addcontentsline{toc}{chapter}{Abstract}

% Set page layout to plain (only page number) and two-column 
\begin{multicols}{2}
\thispagestyle{plain}
\par{
    \textit{
        Medical professionals use \acrfull{mri} or \acrfull{ct} scans as essential components for medical diagnosis, following the course of medical conditions and the planning of medical procedures.
        There is a trend towards machine vision to support medical professionals interpreting and using these images.
        Building these applications requires expensive labelled datasets.
        This research investigates techniques to reduce the dataset labelling cost by working with point annotation instead of full annotation.
        Experiments are conducted on publicly available datasets and demonstrate two new loss components and a combination technique of different model results to generate pseudo masks.
        As a final result, this work demonstrates that one can obtain 72 \% of the performance of a fully annotated model at an estimated 24 \% of the labelling cost. 
    }
}
\section*{Thesis objective \& Motivation}
\par{
    The use of radiological images is a crucial element in modern medical practice. 
    \acrshort{mri} or \acrshort{ct} scans are essential components for pre-operative and post-operative diagnosis, following the course of medical conditions and the planning of medical procedures.
    Automated interpretation of medical images can mean a gain in efficiency.
}
\par{
    Machine vision - deep learning in general - tends to be very \textit{data-hungry}. Constructing a new model requires large, labelled datasets.
    Acquiring these datasets and the corresponding labels is time-consuming and expensive. 
    Maximisation of the return of a given data and labelling budget through is a goal shared by all \acrshort{ml} practitioners.
    The use of weak labels, or sometimes called \textit{hints}, is one approach to attempt this.
    This approach aims to train a model capable of inferring more informative results than the information level explicitly available in the labelling.
}
\par{
    This project presents a model for the automated segmentation of the lumbar vertebrae of the human spine based on point level annotated medical scans.
    Point level annotation is faster and cheaper than providing a complete label mask (estimated at 8.4\% of cost\cite{Bearman2015}), this technique provides a cost-benefit. 
    The labels only contain the true class of a mere handful of voxels. This is a weak label to classify all voxels.
}



\section*{Data sets and data preprocessing \label{sec:abstr_data}}
\par{
    All datasets used in this work are publicly available (all datasets are listed on page \pageref{sec:datasets}). 
    These datasets contain both \acrshort{ct} and \acrshort{mri} scans. 
    In 86 of these scans, complete volume masks of the vertebrae are available. 
    In 20 volumes, only semantic labels are available.
    For 125 volumes, point level annotation is available.
}
\par{
    The complete dataset of 231 patients consists of 112 women and 99 men. Of 23 people, no gender information is available. 
    Since a medical professional does not order a medical scan unless there is a suspicion of a medical condition, the dataset contains various patients with different pathologies,
    such as patients with scoliosis and with crushed and wedged vertebrae.
}
\par{
    Different datasets vary in data formats and different scan resolutions. 
    Data preprocessing starts with homogenising the scan resolution by resampling the image on an $1mm\times 1mm\times 1mm$ grid. 
    Next, the image is sliced along one of the three principal axes.
    The contrast of the 2D image slices is first enhanced with the \acrfull{clahe} algorithm.
    Then the images are cropped (or padded, if needed) to form $352 px \times 352 px$ slices.
    All models are built with this image size, sufficient to contain all 5 lumbar vertebrae $L_1$ to $L_5$ in one image.
}


\section*{Methodology}
\par{
    The performances of different models are compared based on the class-weighted dice score.
    This metric takes into account both the model precision and recall as well as the class imbalance.
}
\par{
    For 86 scans, full annotation masks are available.
    As a performance benchmark, the performance of a fully supervised model trained on these images ($Dice_w=0,76$) is taken.
}
\subsection*{Weakly supervised models}
\par{
    The model backbond is the VGG16-FCN8 network, pre-trained on a large classification dataset.
    The model estimates 6 segmentation classes (5 lumbar vertebrae and the background class). 
    By training three different weakly supervised models on sets of 2D images sliced along the 3 main volume dimensions, three sets of segmentation masks are obtained. 
    The combination of these different segmentation masks is used as an \textit{pseudo} label set to train a fully supervised model on one volumetric dimension.
}

\subsubsection*{Loss function}
\par{
    To train the weakly supervised network, several loss components, both supervised and unsupervised, are combined.
    The model loss to train three point-supervised models in the first step of the procedure presented in this work consistents of 4 components:
    the point loss $\mathcal{L}_P$ and the consistency loss $\mathcal{L}_C$ were defined in \cite{Laradji2021} by I. Laradji, while this work introduced the prior extend and separation loss components $\mathcal{L}_E$ and $\mathcal{L}_S$ are introduced in this work.
}
\par{
    The weighted cross-entropy loss is optimised for the fully supervised reference model, a classic choice for this problem.
    It is also the point loss $\mathcal{L}_P$ component of the weakly supervised model. Then it is only evaluated on the set of available point labels $\mathcal{I}_i$.
    The function combines the six network output channels with a softmax function $\sigma$, after which the negative log-loss function is calculated, weighted with factors $w$.
}
\begin{equation} \label{eq:crossEntropy}
    \mathcal{L}_P(X_i) = -\sum_{\vec{p} \in \mathcal{I}_i} w_{\mathcal{Y}_i(\vec{p})}.\log\left[\sigma_{\mathcal{Y}_i(\vec{p})}\left(\vec{z_i(\vec{p})}\right)\right]
\end{equation}
\par{
    The unsupervised rotation consistency loss $\mathcal{L}_C$ imposes that the model output $f_\theta$ should be consistent for a transformation $t_k$ of the input image.
    In this work, the chosen transformations are image rotations over $0^\circ, 90^\circ, 180^\circ$ or $270^\circ$, combined with an image flip.
}
\begin{equation}
    \mathcal{L}_C(X_i) = \sum_{p \in \mathcal{P}_i} \left| t_k\left[f_\theta(X_i)\right]_p - f_\theta\left( t_k[X_i] \right)_p  \right|  
\end{equation}
\par{
    The second unsupervised loss term is the separation loss term. 
    Due to the low volume of labelled voxels, the the model lacks the incentive to output differentiating expressions of the output channels $\vec{z}_i$.
    $\mathcal{L}_S$ forces the model to do this.
}
\begin{equation}
    \mathcal{L}_S(X_i) = - \sum_{\vec{p}} \sum_{m\in \mathcal{K}} \sum_{n \in \mathcal{K}, n>m} \mathbf{S}(z_i[m]) - \mathbf{S}(z_i[n])
\end{equation}
\par{
    Finally, $\mathcal{L}_E$, the maximal extend supervised loss term, takes into account that a lumbar vertebra has a limited size ($r=110mm$).
    The Euclidian distance field $\mathbf{d}$ from the annotation point is converted to a semi-mask for each class $k$:
    \begin{eqnarray}
        \mathbf{d}_k(\vec{q}) &=& \max_{\vec{p}:\mathcal{Y}_i(\vec{p})=k}||\vec{q} - \vec{p}||\\
        \mathbf{m}_k(\vec{q}) &=& \mathbf{I}\left( (-\mathbf{d}(\vec{q}) + r) > 0 \right)
    \end{eqnarray}
    Now, $\mathbf{m}$ is 1 only for positions closer than distance $r$ from the annotation points for class $k$.
    Where $\mathbf{m}_k=0$, the model output should not indicate output class $k$. Where $\mathbf{m}_k=1$, the output class is unknown.
    The loss function is the binary cross-entropy between $\mathbf{m}_k$ and the sigmoid of the k$^{th}$ channel of the logits $z_i$ with weight vector $\{1, 0\}$.
}
\begin{equation}
    \mathcal{L}_E(X_i) = \sum_{k\in\mathcal{K}}\sum_{\vec{q}\in X_i}  (1-\mathbf{m}_k(\vec{q})) \log(\mathbf{S}(z_i(\vec{q})_k)) 
\end{equation}

\subsubsection*{Model result combination}
\par{
    Combining the results of the three models trained on the three geometric axes (transverse, frontal \& sagittal) is a pseudo-mask of higher quality than the results of the individual models.
    After morphological smoothing, the pseudo mask is used to train the final model (on sagittal slices).
}
\thispagestyle{plain}
\section*{Results}
\subsection*{Hyperparameter optimization}


\subsection*{Final result}

\section*{Conclusion}
\todo[inline]{Complete this section}
\cleardoublepage
\end{multicols}
\newgeometry{total={210mm,297mm},left=20mm,right=20mm,bindingoffset=5mm, top=25mm,bottom=25mm} 
\begin{partwithabstract}{Problem Introduction \& Motivation}
    The objective of this project is the development of a model for the automated vertebrae instance segmentation of \acrshort{mri} and \acrshort{ct} scans of the lumbar part of the human spine based on point annotated training data.
    
    This part of the book aims to clarify these terms. 
    I start off with some information regarding the human spine and the medical imaging techniques to investigate it.
    Secondly, I define and clarify different machine learning problems and techniques when working with two and three dimensional images.
    This part of the book ends with a discussion of previous work. 
    First the previous work to tackle the problem of instance segmentation of the human spine from medical images.
    Then prevous work in the field of \Gls{weaklysupervisedl} \Gls{machinevision}.
\end{partwithabstract}

\restoregeometry
\chapter{The human spine}
\par{
    This document presents a model for automatic segmentation of \acrfull{ct} and \acrfull{mri} images of the lumbar spine.
    This chapter introduces these terms\footnote{This work is not a medical desideration. For in-depth knowledge on the anatomy and physiology of the spine, consult the specialized literature.}.
    Secondly, it gives a basic overview of the medical imaging techniques. 
    Finally, it touches upon a specific medical procedure in which this information is used: the minimally invasive surgery of a spinal hernia.
}



\section{Anatomy of the human spine}

\todo[inline]{Assure all citations are ok.}
\par{
    The spinal column, vertebral column or backbone \footnote{NL: \textit{wervelkolom}} is a structure of 34 bones. 
    It holds the body upright while providing it with the mobility to bend and twist.
    the \textit{intervertebral discs} make this possible. These consist of a ring of fibrocartilage and an inner gel-like centre and form an articulation between two vertebrae.
    Moreover, the vertebral column serves as a conduit for significant nerves running from the brain to the toes.
    The spinal column, as illustrated in figure \ref{fig:spineimage} can be divided into five regions:
}

\begin{description}
    \item[the Cervical spine:] 7 vertebrae of the neck, indicated by C$_1$ to C$_7$\footnote{Vertebrae are numbered from the head down. C1 (\textit{atlas}) is the vertebra closest to the head}.
    \item[the Thoracic spine:] 12 vertebrae of the middle back (T$_1$ to T$_{12}$).
    \item[the Lumbar spine:] 5 vertebrae that form the lower back. These are commonly referenced as L$_1$ to L$_5$.
    \item[the Sacrum:]\footnote{NL: \textit{Heiligbeen}} This is a structure consisting of 5 naturally fused vertebrae (S$_1$-S$_5$).
    \item[the Coccyx:]\footnote{NL: \textit{Stuit of staartbeen}} Structure of 3 to 5 naturally fused vertebrae at the end of the spinal column.
\end{description}


\begin{SCfigure}[][h!]
    \centering
    \includegraphics[width=10cm]{/home/thesis/images/SpineModel.jpeg}
    \caption{\label{fig:spineimage}Model of the human spine. The five vertebrae in green form the lumbar spine. They are referred to as \textit{L1} to \textit{L5} from top to bottom. }
\end{SCfigure}

\section{Pathologies of the human spine}
\par{
    This document does not aim to provide an exhaustive list of all human spinal pathologies. 
    Two pathologies are interesting to discuss further as an introduction to this work since they occur in the data used in this project, see page \pageref{sec:datasets} for further details.
}
\par{
    First, there is \textit{scoliosis}, which is a sideways curve of the spine.
    The severity of this condition can vary from relatively mild to severe. 
    In severe cases, scoliosis can affect the patient's movement and breathing.
    \todo[inline]{image}.
}
\par{
    Second, there is the \textit{spinal hernia} or spinal disc herniation\footnote{Spinal disc herniation is sometimes referred to as a \textit{slipped disc}.}. 
    A spinal hernia is caused by damage to the \textit{annulus fibrosus}\footnote{The fibrocartilage ring around the softer gel-like centre of the intervertebral disc.}. 
    This damage can cause the intervertebral disc to bulge out. 
    This condition is painful due to the inflammation reaction, and in severe cases, the bulging material can irritate or cause impingement of the critical nerves along the spine.
    The nerve impingement can even lead to radiating pain to the limbs or even limb paralysis.
    In severe cases, a spinal hernia requires surgical treatment.
    This specialized procedure requires repeated medical imaging to allow the surgeon to accurately investigate the situation before the operation and assure correct positioning of the instruments during the intervention.
}
\begin{SCfigure}[][h!]
    \centering
    \includegraphics[width=10cm]{/home/thesis/images/REISS_illustration.png}
    \caption{Artist's impression of the start of the surgical treatment of a lumbar hernia. 
    This image shows the dilator through which the surgical instruments will be inserted to mill away the bulging disc material.
    This procedure requires repeated visualization of the instruments and the spine via X-ray images. \textit{This illustration is designed Verhaert NP\&S}.\label{fig:REISS_procedure}}
\end{SCfigure}
\par{
    In image \ref{fig:REISS_procedure}, the start of the surgical treatment procedure for a lumbar hernia is illustrated.
    One possible benefit of improved automatic interpretation of \acrshort{mri} and \acrshort{ct} scans is providing support for this procedure.
    This delicate procedure requires interpretation and combination of information from both scan types (at different times). This is a demanding task.
}

\section{Medical imaging of the human spine\label{sec:medical_imaging}}
\marginpar{
        \includegraphics[width=5cm]{/home/thesis/images/MRI_CT_images.jpeg}
        \captionof{figure}{Illustration of \acrshort{ct} and \acrshort{mri} images of a human lumbar spine.}
        \label{fig:mri_ct}
    }
\par{
    For diagnosis and visualization of spinal pathology several medical imaging techniques are used: \acrfull{ct}, \acrfull{us} and \acrfull{mri}. 
    These techniques allow non-invasive visualization of the spine and discs in three dimensions.
    Two dimensional slices from volumetric scans (\acrshort{mri} and \acrshort{ct}) are illustrated in \ref{fig:mri_ct}.
}

\subsection{CT scan}
\par{
    The \acrfull{ct}\footnote{Also known as CAT-scan, for Computed Axial Tomography or Computer-assisted Tomography.} is a non-invasive medical imaging procedure
    \footnote{CT scans are primarily used for medical purposes. 
    It is also used in industry for non-destructive component and assembly inspection.
    In geology, it is used to identify materials in a drill core quickly. In archaeology, it is used for the non-destructive investigation of artefacts. } 
    for diagnostic purposes. 
    A \acrlong{ct} procedure consists of the combination of an array of X-ray attenuation images taken with a rotating X-ray tube as illustrated in figure \ref{fig:CT_principle}. 
    These images can be combined with a tomography algorithm to reconstruct a volumetric representation of the radiographic density.
}
\par{
    \acrshort{ct} is a versatile technique with various medical diagnostic applications. 
    Contrary to \acrfull{mri} imaging, this technique is suitable for patients with a pacemaker or insulin pump since there are no magnetic fields involved.
    The main disadvantage of \acrshort{ct} is the exposure to ionizing radiation of the patient and the risk of exposure of the medical professional to the same radiation.
    The image quality increases with radiation dose, but so does the probability of radiation-induced cancer.
    Improving the reconstruction algorithms to obtain higher-resolution images without reducing the radiation dose is an ongoing area of research. 
}
\marginpar{
        \includegraphics[width=5cm]{images/FDA_CT_Scan.png}
        \captionof{figure}{Conceptual illustration of the working of a \acrlong{ct} scan. Image from \url{www.fda.gov/radiation-emitting-products/medical-x-ray-imaging/computed-tomography-ct}}
        \label{fig:CT_principle}
    }

\subsection{MRI scan}
\par{
    The \acrfull{mri} scan is a medical imaging technique that is not based on ionizing radiation\footnote{Eventhough the alternative name \textit{nuclear} magnetic resonance (NMR) might confuse.}.
    \acrshort{mri} imaging will visualize the concentration of hydrogen\footnote{In theory, other atoms than hydrogen can be excited by adapting the excitation frequency. This is rarely done.} atoms.
    The patient is positioned in a tunnel where a high (up to several Tesla) constant magnetic field is applied. A temporary oscillating signal is applied with the resonance frequency corresponding to hydrogen atoms is superposed on the static magnetic field.
    The hydrogen atoms will fall back to the equilibrium state, emitting radiofrequency (RF) signals. These can be measured by receiving coils.
    After excitation, the hydrogen in the water atoms in the patient's body tissues returns to equilibrium. \acrlong{mri} is particularly suitable for visualization of tissue with higher water content, such as tumours and infections, and to visualize fat.
}
\par{
    An \acrshort{mri} scan can be \textit{T1 weighted} or \textit{T2 weighted}. 
    The difference between these two techniques lies in whether the image is constructed based on the relaxation time of the magnetization is colinear with the direction of the static field (T1) or the magnetization perpendicular to the static field (T2).
    While areas with higher water content, such as infected areas, will release a higher signal on a T2-weighted \acrshort{mri} scan, the same images will show a lower signal strength for T1-weighted scans.
    The use of these different techniques of \acrshort{mri} scans is application-dependent.  
}
\par{
    Contrary to \acrfull{ct} scans, the \acrlong{mri} procedure does not expose the patient of radiation. Due to the high magnetic fields, the technology cannot be used for patients with pacemakers, cochlear implants or other metallic objects in the body.
    \acrshort{mri} allows to visualize soft tissue better than \acrshort{ct} images. An \acrshort{mri} image allows visualizing both the grey and white brain matter, while this is not possible with \acrshort{ct} images.
    Although both techniques produce images that resemble each other, none of both techniques can replace the other one completely.
}


\chapter{Artificial intelligence \& Machine Learning}

Machine learning is a branch of \Gls{ai}. 
The field of \Gls{machinevision} is a branch of machine learning.
  Some impressive steps forward have been made in the past decade, mainly thanks to the emergence of \Gls{deepl}.

The aim of this chapter is not to provide a complete overview of the machine learning field.
Instead, the objective is to highlight concepts meaningful for this work.

\section{Artificial Intelligence}

The field of \Gls{ai} (AI) is the engineering discipline of the automation of \textit{cognitive} tasks.
Tasks such as search, control and classification are generally considered to require a level of intelligence. 
Automation of this type of tasks to allow a machine to perform them is thus \Gls{ai}\footnote{To be precise, it is not the human intelligence that is replicated. It is the \textit{effect} of this intelligence.}.
A classic PID controller and even a bang-bang (thermostat) controller can be viewed as simple but very effective forms of AI.


This engineering discipline has advanced incredibly in the past decades, driven both by leaps forward in the available hardware and new algorithms and models. \\
First, the availability and reliability of hardware components such as sensors, cameras, digital storage and calculation power have increased exponentially 
\footnote{ \textit{Moore's law} states that the number of transistors on an integrated circuit doubles every two years. This rate of progress has held more or less for a wide range of digital components in the past decades.}.
The size and price of these components decreased equally dramatically. \\
Second, progress has been made in developing algorithms to use this available data and computation power to solve problems and perform tasks.
This text does not provide a complete overview of all existing machine learning models. 
This work makes use of \Gls{deepl} models.
A \Gls{deepl} model is a type of \acrfull{ann} with multiple hidden layers. 
Deep learning models are behind almost all modern applications of \acrfull{nlp} and \Gls{machinevision}.

An \acrshort{ann} is a collection of connected nodes, or \textit{neurons}. 
These are typically structured in layers. 
There is always an \textit{input} layer and an \textit{output} layer. Between these two, one can find the \textit{hidden} layers\footnote{When there is at least one hidden layer, one talks about a deep network. In practice, \acrshort{cnn}s have multiple hidden layers.}.
To calculate a node value, the incoming node values are weighted by the connection weights and the result of this linear combination is transformed by an activation function\footnote{
    $\vec{x}$ : values of the nodes connected to node $j$\\
    $\vec{w}$ : connection weights\\
    $\sigma(.)$ : activation function
}:
\begin{eqnarray}
    z_j(\vec{x} | \vec{w}) &=& \frac{\vec{w}^T\vec{x}}{\sum w_k} \\
    y_j(\vec{x} | \vec{w}) &=& \sigma(z_j)
\end{eqnarray}
Deep learning models can be fitted through the \acrfull{bp} algorithm.

\todo[inline]{Short and to-the-point introduction of neural networks}

Constructing a network requires evaluating it, comparing the evaluation output to a known desired output, and taking steps to bring the model output closer to the desired output. 
The procedure used to 
A network is fitted to the \textit{train set} by the optimization algorithm, based on the \textbf{loss}.
The \acrshort{ml} engineer wants to judge the performance of the model based on one or several \textbf{metrics}, calculated on the \textit{train set}, the \textit{cross-validation set} and eventually on the \textit{test set}.

\section{Machine vision}

\Gls{machinevision} is the branch of \Gls{ai} focussed on image processing.
The machine vision task performed in this work is called instance \Gls{segmentation}.
This chapter explains what this means. 
The segmentation task is compared to other machine vision tasks.

This work investigates the use of \Gls{weaklysupervisedl} data for the training and segmentation model. 
The concept and benefits of \Gls{weaklysupervisedl} machine learning is discussed.

\subsection{Machine vision tasks \label{sec:machinevisiontasks}}

\Gls{machinevision} is a broad discipline. 
Humans extract information from images almost subconsciously, and we are often not aware of the different tasks we perform on images.
The objective of this section is to briefly define different machine vision tasks discussed further in this book. 
Several machine vision tasks consist of \textit{recognizing} objects, animals or humans in an image.
A model is built for a finite list of \textit{categories} that can be present in an image.
Depending on the question asked ad inference time, one can distinguish the following tasks.

\begin{description}
    \item[Image classification] is the task of determining what object category\footnote{or categories} is present in the image. Is there a cat in this image?
    \item[Object counting] is the task of counting how many instances of each category are in the image. How many cats are there in this picture? 
    \item[Object detection] consists not only of identification of the object. Also, the spatial position is requested, often in the form of a bounding box. Where is the cat in this picture if a cat is present?
    \item[Semantic segmentation] requires a class estimation for each image pixel. Pixels that do not belong to a specific class are called the \textit{background}.
    \item[Instance segmentation] requires not only that the semantic class is determined for each pixel, but also that two individuals of the same class\footnote{say, two cats.} are distinguished.   
\end{description}

Figure \ref{fig:machinevisiontasks} illustrates the difference between these machine vision tasks. 

\begin{SCfigure}[][h!]
    \centering
    \includegraphics[width=10cm]{/home/thesis/images/Classification_vs_Segmentation.jpg}
    \caption{Illustration to compare different Machine vision tasks \cite{SemTorch76:online}. 
    Object detection means that the location of several objects is estimated by the model. This is indicated by the \textit{bounding boxes}.
    Segmentation of an image is classifying each pixel in the correct class or assigning it to the \textit{background} class.
    Semantic segmentation makes no difference between different instances of the same semantic class, instance segmentation does.
    \label{fig:machinevisiontasks}}
\end{SCfigure}

Other interesting applications of \gls{machinevision} include\footnote{This list is not exhaustive.}:
\begin{description}
    \item[Face recognition] is the identification of human faces. 
    \item[Image reconstruction] or \textit{inpainting} consists of recreating parts of a damaged image.
    \item[Image captioning] consists of the creation of full sentences describing the content of an image.    
\end{description}

\subsection{The convolution layer}

The building block of virtually every modern machine vision network, including the ones in this thesis, is the convolution layer.

\todo[inline]{weight sharing, feature extraction}

\subsection{Data for training machine vision models}

To perform the tasks discussed in chapter \ref{sec:machinevisiontasks}, one needs to build a suitable model.
For \Gls{machinevision} tasks\footnote{and many other tasks.}, the current standard approach is \Gls{deepl}.


The cost to generate, store and communicate images and computation power has dropped in the past decades.
This evolution allows to train a model  on previously unimaginable quantities of data\footnote{The \textit{ImageNet} database (\url{http://image-net.org/challenges/LSVRC/index}) consists of more than $14.10^6$ images.}.
This technique allows a model with a high number of degrees of freedom to be trainded\footnote{learn by example, so you will} without the need for expert-crafted features. 


Collection of this dataset - most importantly, the labels - proves to be a challenge. 
This chapter tries to explain what \textit{weak labels} and \textit{strong labels} are and what the difference is between both.

\subsubsection{Training a model}

To build a model to perform the tasks discussed in \ref{sec:machinevisiontasks}, this model needs to be trained.
This requires a set of \textit{labelled} images. 

In the classic approach, the supervision type closely resembles the intended model output.
To train a model that can classify an image\footnote{Given an image, the model outputs if this picture is a representation of class \textit{cat}, \textit{dog} or another animal or object. }, 
one has to \textit{train} the model on a set of labelled images where a human indicated the class.
To train a model to perform image segmentation\footnote{segmentation means that the model classifies each pixel.}, an expert needs to provide a set of images in which
each picture, the objects are delineated, indicating their class.  



\subsubsection{Weak supervision types}


Building a machine vision model requires a collection of images where an expert in the intended task has provided correct information from which the model can \textit{learn}.
Depending on the model objective, other types of labelling are required.


Figure \ref{fig:ImageLabelTypes} illustrates several types of image supervision : 
From left to right on the top row, this shows point supervision and squiggle annotation. ON the second row, bounding box annotation and complete mask annotation are illustrated.
The generation of these labels is costly and time-consuming.
Especially \gls{deepl} models are known to be very data-hungry. 

\begin{SCfigure}[][htb]
    \centering
    \includegraphics[width=10cm]{/home/thesis/images/McEver.png}
    \caption{Four different annotation types \cite{McEver2020}: 
    On the top left the picture is point level annotated. The points are inflated for visibility.
    On the top right, squiggle annotation is used.
    The bottom left shows bounding box supervion.
    While the bottom right image is fully annotated.
    An image level label would indicate that there are multiple instances of \textit{person} and \textit{bike} in the image.
    \label{fig:ImageLabelTypes}}
\end{SCfigure}

Several researchers have investigated ways to train computer vision models with cheaper labels, given the high labelling cost.
This branch of research is known as \Gls{weaklysupervisedl}.
The objective is to construct a robust model based on \textit{cheap} (incomplete, noisy or imprecise) labels, sometimes described as \textit{indirect supervision}.
Numerous creative approaches have been conceived. 
It is impossible to give an exhaustive list of approaches. 
In what follows, I will mainly focus on the approaches I chose to investigate myself, but I will also try to give some hints of the remarkable creativity found in the field.
Since the provided annotations in \Gls{weaklysupervisedl} are not full labels, these are sometimes described as \textit{hints} instead\footnote{
    This is based on the insightfull talk at \url{ 
        https://youtu.be/4EjYxVVCAaE
    }. For example, the destinction between labels and hints.
}.
The basic concept of \Gls{weaklysupervisedl} is that there are two sources of information to draw from: The hints and the prior knowledge about the problem (Priors).
These \textit{Priors} can be any form of prior knowledge about the object to be segmented\footnote{or any other machine vision task.}.
Priors can be the object size, shape or location, the number of instances, the similarity across images or the similarity with external images.

Whether an annotation is considered a \textit{weak label} or a \textit{strong label} depends more on the modeller's intention than on the annotation itself. 
When one aims to construct a model to infer output labels with a higher informative value than the original annotations, these \textit{labels} become \textit{hints}.
Making a model predict bounding boxes from a dataset annotated with bounding boxes means considering these as \textit{strong labels}. 
If one uses the same dataset to construct a model that predicts pixel-wise masks, the labels are \textit{weak labels} or \textit{hints}.

For a segmentation task, weak labels can be:
\begin{description}
    \item[Image level labels]: In this case, only the object class of the object in the image is provided. 
    This would be a full label for a classification task, but it is a weak label for a segmentation task.
    \item[Point annotation]: This annotation technique, the subject of this work, consists of asking the expert to indicate the classes with one or several points. This technique is used by Laradji
    \item[Squiggles]: This annotation technique is related to the point annotation technique. Instead of points, the expert is asked to indicate the classes with a squiggly line.
    \item[Image description]: This task combines the problem of \acrlong{nlp} with the problem of image segmentation. The annotation of the image is derived from verbal or written description of the image. 
    This annotation type has not yet been used by many researchers. 
    It might be promising since large bodies of datasets could be available from for example medical files where a medical expert has provided a written diagnosis based on available medical images. 
\end{description}

\todo[inline]{Motivation of weakly supervised learning --> Difference in annotation time and cost from Bearman and Laradji Covid}
\chapter{Previous work}

This project sits at the intersection of two areas of research. 
First there is the application of \Gls{ai} in medical applications, specifically for segmentation problems.
Then, there is the active area of research of \Gls{weaklysupervisedl} machine learning.


Specifically I build strongly upon two existing results:

\begin{enumerate}
    \item The U-Net based approach for building a vertebra instance segmentation model by dr. N. Lessmann et al. \cite{Lessmann2018} based on fully supervised data. This model was developed further in \cite{Chuang2019}
    \item The \acrfull{wise} approach developped by dr. I. Laradji \cite{Laradji2020,Laradji2018}. 
\end{enumerate}

\section{Artificial intelligence for medical applications}

\todo[inline]{Start with short overview of other AI problems (1/2 page)}

\Gls{ai} has proven to be a valuable contribution to medical practice to reduce the burden of repetitive tasks on the medical caregiver.

\todo[inline]{elaborate: ppg to blood pressure - slaapapnue}

\section{Segmentation problems for medical applications}

\todo[inline]{General introduction of U-Net and other medical approaches}

For segmentation tasks, the U-net \cite{Ronneberger2015} is widely used. 
This architecture can be represented by a characteristic U-shape, as the name indicates.
It consists of 

\begin{SCfigure}[][htb]
    \includegraphics[width=10cm]{/home/thesis/images/UNet_Ronneberger.png}
    \caption{U-Net architecture, as illustrated in \cite{Ronneberger2015}. 
    Each blue box represents a multi-channel feature-map. 
    The number of channels is indicated above the box, the $x \times y$ dimensions are indicated at the bottom left.
    The gray arrows indicate the feature maps in the contracting path are copied and concatenated to the feature maps of the expanding path.}
    \label{fig:unet}
\end{SCfigure}

\todo[inline]{General introduction of U-Net and other medical approaches}

\subsection{Segmentation of the human spine}

\todo[inline]{other authors, approaches --> priors used, metrics used, datasets used}

In this work, the network developed in \cite{Lessmann2018} is used.

\todo[inline]{Concept behind Lessmann, datasets used and results}

\section{Weakly supervised segmentation}

\subsection{General approaches}

\todo[inline]{How is this problem generally solved: PCAMS, WISE, ...}

In this work, the \acrshort{wise} concept is applied for vertebra segmentation.


\begin{SCfigure}[][htb]
    \includegraphics[width=10cm]{/home/thesis/images/Laradji_architecture.png}
    \caption{Illustration from \cite{Laradji2020}. The \acrshort{wise} approach consists of two branches: The Embedding branch and the Localization branch.}
    \label{fig:Laradji}
\end{SCfigure}

\subsection{Weakly supervised segmentation for Medical applications}

Laradji COVID 19


\newgeometry{total={210mm,297mm},left=30mm,right=30mm,bindingoffset=5mm, top=25mm,bottom=25mm} 
\begin{partwithabstract}{Modelling \& Analysis}
    This part of the document details the used datasets and the modelling and evaluation methodology.
\end{partwithabstract}
\restoregeometry

\chapter{Datasets}

\section{Dataset origin and characteristics}
\todo[inline]{Overview of the datasets gathered. MRI / CT, number of images, type of annotation}

\section{Comparison of the different datasets}

Patient Comparison: Gender, age, health and pathologies in the spine.

Image comparison: 
Distribution of image intensities. Histogram of intensities.
Information about original image grid shape.


\chapter{Proposed methods}

\section{Fully supervised model}

\subsection{Model}

Lessmann, discussed in xxx.
The specific implementation is based on github repo xxx.

\subsection{Data}

Discussion on approach for using the datasets I have with the Lessmann model.
How to preprocess the different datasets?
What data augmentation steps used?

\subsection{Hyperparameters}

Hyperparameters for model and training from \cite{Lessmann2018}

\section{Weakly supervised model}

\subsection{Model}

Model is based on Laradji \cite{Laradji2020}

\subsection{Data}

2D model -> volumes are sliced.
Discussion of this procedure

\subsection{Hyperparameters}



\chapter{Experiments}

\section{Experimental results of the fully supervised model}

\section{Experimental results of the weakly supervised model}

\section{Comparison between fully \& weakly supervised model}

\section{Final model proposal}

\section{Experimental results of the final model}

\chapter{Conclusion}

Comparison between results with weakly supervised model and fully supervised model.

\appendix

\newgeometry{total={210mm,297mm},left=30mm,right=30mm,bindingoffset=5mm, top=25mm,bottom=25mm}
\begin{partwithabstract}{Appendix}
  Apendices to the document:
  \begin{enumerate}
    \item Software environment used
    \item Agreement documents for use of two datasets.
  \end{enumerate}
\end{partwithabstract}
\restoregeometry
\chapter{Medical terms}

Clarification of the medical lingo used.

\begin{SCfigure}[][htb]
  \centering
  \includegraphics[width=10cm]{/home/thesis/images/Anatomical_Planes.png}
  \caption{Clarification of the terms regarding the anatomical planes
  }
\end{SCfigure}

\begin{SCfigure}[][htb]
  \centering
  \includegraphics[width=10cm]{/home/thesis/images/prone_supine.png}
  \caption{Supine (face-up) and Prone (face-down) position of a patient.}
\end{SCfigure}

\chapter{Used software}

\todo[inline]{Software environments can be build with dockerfiles available in folder \textit{/dockerfiles/code/} of the git.}

\todo[inline]{Assure proper referencing of all libraries used}

\begin{SCtable}[\sidecaptionrelwidth][h]
 
  \begin{tabular}{ p{6cm} l l } 
   \hline
   \hline
   Library & version & reference  \\
   \hline 
   PyTorch & 1.7.1 &  \\ 
   SimpleITK &  &  \\ 
   \hline
   \hline
  \end{tabular}
  \caption{Python libraries used}

\end{SCtable}
\newgeometry{total={210mm,297mm},left=30mm,right=30mm,bindingoffset=5mm, top=25mm,bottom=25mm}
\chapter{Dataset agreements}

\includegraphics[width=17cm]{/home/thesis/images/AgreementxVertSeg.png}

\printbibliography

\end{document}


\subsection{PLoS Dataset}

The \textit{PLoS} dataset was compiled for \cite{Chu2015} in 2015 by dr. C. Chu, University of Bern, Bern, Switzerland and made publically available \footnote{See : \url{ http://doi.org/10.5281/zenodo.22304 }} .
It consists of 23 T2-weighted spine \acrshort{mri} scans. 
Contrary to other datasets, the segmenation labels in this dataset do not provide information to destinguish the individual vertebrae from each other.

\subsubsection{Original Objective of the Dataset}

In \cite{Chu2015} the development of a random forest regression approach for spine vertebrae segmenation and classificiation is described.
The results of several random forest regressors and classifiers is unified with a voting mechanism.
This approach obtains a mean Dice metric score of 88.7\%.

\subsubsection{Patient statistics}

Due to the anomimization process, the \textit{PLoS} dataset does not contain patient information.
This means that it is not possible to produce any statistics regarding patient age or gender.

\subsubsection{Technical information}

As is indicated in figure \ref{fig:AllDataset_dims}, and in figure \ref{fig:PLoS_img02}, the PLoS image volumes are cropped in the left-right direction.
The volumes are consistently $381mm \times 381 mm \times 78 mm$, where the shortest dimension is in the left-right direction.

\begin{SCfigure}[][htb]
    \centering
    \includegraphics[width=.95\textwidth]{automated_graphs/PLoS_img02.png}
    \caption{
        PLoS dataset scan \textit{img02}. \label{fig:PLoS_img02}. The craniocaudal direction is cropped in a way comparable to the volumes in the USiegen dataset, but, the direction of this axis is inverted.
        The left-right axis is cropped similar to the USiegen data volumes.
    }
\end{SCfigure}
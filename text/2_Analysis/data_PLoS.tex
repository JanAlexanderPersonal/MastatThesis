\subsection{PLoS Dataset}

The \textit{PLoS} dataset was compiled for \cite{Chu2015} in 2015 by dr. C. Chu, University of Bern, Bern, Switzerland and made publically available \footnote{See : \url{ http://doi.org/10.5281/zenodo.22304 }} .
It consists of 23 T2-weighted spine \acrshort{mri} scans. 
Contrary to other datasets, the segmentation labels in this dataset do not distinguish the individual vertebrae from each other.

\subsubsection{Original Objective of the Dataset}

In \cite{Chu2015} the development of a random forest regression approach for spine vertebrae segmentation and classification is described.
The results of several random forest regressors and classifiers are unified with a voting mechanism.
This approach obtains a mean Dice metric score of 88.7\%.

\subsubsection{Patient statistics}

Due to the anonymization process, the \textit{PLoS} dataset does not contain patient information.
This means that it is not possible to derive any statistics regarding patient age or gender.

\subsubsection{Technical information}

As is indicated in figure \ref{fig:AllDataset_dims}, and in figure \ref{fig:PLoS_img02}, the PLoS image volumes are cropped in the left-right direction.
The volumes are consistent $381mm \times 381 mm \times 78 mm$, where the shortest dimension is in the left-right direction.

\begin{SCfigure}[][htb]
    \centering
    \includegraphics[width=.95\textwidth]{automated_graphs/PLoS_img02.png}
    \caption{
        PLoS dataset scan \textit{image002}. \label{fig:PLoS_img02}. The craniocaudal direction is cropped in a way comparable to the volumes in the Siegen dataset, but the direction of this axis is inverted.
        The left-right axis is cropped, similar to the Siegen data volumes.
    }
\end{SCfigure}
\begin{SCfigure}[][htb]
    \centering
    \includegraphics[width=.95\textwidth]{images/PLoS_s8_mask.pdf}
    \caption{
        PLoS dataset scan 008 saggital slice, compared to the ground truth mask.
        The Gls{groundtruth} masks for this dataset do not include the vertebra laminae, only the vertebra body.
        The mask only contains one class. No distinction between different lumbar laminae is available.
        \newline\noindent Colour legend: \newline
\noindent\mycircle{colL1}  L1, L2, L3, L4 \& L5
    }
\end{SCfigure}
\chapter{Abstract}

\section{Thesis objective}
In this thesis, I am to develop a model for the automated instance segmentation of the vertebrae of the human spine based on point level annotated data.
The input images can be generated by by \acrfull{ct} and \acrfull{mri}.


The models are trained based on point-level annotated volumes, nevertheless the model output is an estimation of the instance class of each voxel.
Since point level annotation is faster and cheaper than labelling complete volumes, this technique provides a cost benefit. 

\section{Data \& Methodology}

\subsection{Data}
All datasets used in this work are publicly available. 
The scope of this work did not include medical data gathering.
A list of all datasets used can be found in \hlnote{xxx}{Write appendix}.


These datasets contain both \acrshort{ct} and \acrshort{mri} scans. 
In \hlnote{100}{Update for final version} of these scans, 

\subsection{Methodology}
The performance of the models trained on \Gls{weaklysupervisedl} data is compared to the performance of a model trained on \Gls{supervisedl} data.


\section{Results}
\todo[inline]{Complete this section}

\section{Conclusion}
\todo[inline]{Complete this section}
\chapter{Abstract}

\section{Thesis objective}
In this thesis, I am to develop a model for the automated instance segmentation of the vertebrae of the human spine based on point level annotated data.
The input images can be generated by by \acrfull{ct} and \acrfull{mri}.


The models are trained based on point-level annotated volumes, nevertheless the model output is an estimation of the instance class of each voxel.
Since point level annotation is faster and cheaper than labelling complete volumes, this technique provides a cost benefit. 

\section{Data \& Methodology}

\subsection{Data}
All datasets used in this work are publicly available\footnote{A list of all datasets used can be found in appendix xxx.}. 
The scope of this work did not include medical data gathering.

These datasets contain both \acrshort{ct} and \acrshort{mri} scans. 
In \hlnote{100}{Update for final version} of these scans, complete volume masks of the vertebrae are available. 
For 200 volumes, point level annotation is available\footnotetext{There are also 5 volumes for which no segmentation data is available and 20 volumes for which only semantic labels are available. These were not used in this study.}.

\subsection{Methodology}
The performance of the models trained on \Gls{weaklysupervisedl} data is compared to the performance of a model trained on \Gls{supervisedl} data.
This latter model, trained on fully \Gls{supervisedl} data, is an established model from \hlnote{literature}{cite lessmann}. 
This architecture serves as a reference.
The reference model is trained from random initiation on the part of the dataset for which full volume masks are available\footnote{After train-test split.}.  

\section{Results}
\todo[inline]{Complete this section}

\section{Conclusion}
\todo[inline]{Complete this section}
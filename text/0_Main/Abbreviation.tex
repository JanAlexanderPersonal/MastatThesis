
\newglossaryentry{weaklysupervisedl}
{
        name={Weakly Supervised},
        description={Weakly supervised machine learning where the ground truth labels are only partially available. 
        In the context of image segmentation, this can mean that the labels are only provided at image level or that point level annotation in the image is provided.
        The model is trained on incomplete labels, the desired result remains the complete segmentation of the image.
        Just like in the case of \textit{Fully Supervised Learning} the objective is to model the relationship between an \textit{input} and an \textit{output}. 
        Due to the labels being incomplete, there is a stronger need to identify the internal structure of the data, as is the case for \textit{Unsupervised learning}.
        }
}

\newglossaryentry{segmentation}{
        name={Segmentation},
        description={
                The \textit{segmentation} problem in machine vision consists of the classification of each individual pixel or voxel. 
                The problem of \textit{semantic} segmentation is to detect, for each pixel, the object category it belongs to. 
                \textit{Instance} segmentation digs deeper. It identifies for each pixel the object instance it belongs to.
                The difference is that it differentiates between two objects of the same object category in the picture. 
                }
        }

\newglossaryentry{groundtruth}{
        name={Ground Truth},
        description={
                The \textit{Ground Truth} is a term used in machine learning to indicate the ideal expected result. 
                In the context of Instance Segmentation, the ground truth is the true class of every pixel or voxel. 
        }
        }

\newglossaryentry{unsupervisedl}
{
        name={Unsupervised},
        description={
                In an \textit{Unsupervised} machine learning problem, no labels are present. 
                The aim is not to model the relationship between an \textit{input} and an \textit{output}, the aim is to model the structure of the data.
                \todo[inline]{Add examples of algorithms? KNN, MoG}
                }
}

\newglossaryentry{supervisedl}
{
        name={Supervised},
        description={
                (Fully) Supervised Machine Learning task where target labels are present. 
                The objective of these problems is to model the relationship between an \textit{input} and an \textit{output}.
                \todo[inline]{Add examples of algorithms?}
                }
}

\newglossaryentry{tomography}
{
        name={Tomography},
        description={
                Imaging of a volume through the use of a penetrating wave. 
                Through these waves, a collection of images, called \textit{tomograms}, are produced.
                The mathematical procedure to reconstruct the original volume based on these images is called \textit{tomographic reconstruction}.
                A \acrfull{ct} scan is produced through tomographic reconstruction of several X-ray radiographs.
                }
}

\newglossaryentry{machinevision}
{
        name={Machine vision},
        description={
                The branch of Artificial Intelligence with the objective of invering results from images. 
                In this work, these images can be both two dimensional (\textit{pictures}) as three dimensional (\textit{volumes}).
                }
}

\newglossaryentry{ai}
{
        name={Artificial Intelligence},
        description={
                The study of using computers to automatically perform tasks which once were considered only humans could do.
                This includes, but is not restricted to, interpretation of speech and images. It is often refered to with the acronym \textsc{ai}.
                }
}

\newglossaryentry{deepl}
{
        name={Deep Learning},
        description={
                Deep learning is a branch of Machine Learning where a set of multiple sequential layers is used to progressively extract higher-level features from the raw input data.
                }
}

\newglossaryentry{caml}
{
        name={Class Activation Maps},
        description={
                Technique to identify region of an image \textit{responsible} for the classification result.
                }
}


\newacronym{cnn}{CNN}{Convolutional Neural Network}
\newacronym{crf}{CRF}{Conditional Random Field}
\newacronym{ct}{CT}{Computer Tomography}
\newacronym{mri}{MRI}{Magnetic Resonance Imaging}
\newacronym{ml}{ML}{Machine Learning}
\newacronym{mil}{MIL}{Multi-Instance Learning}
\newacronym{rnn}{RNN}{Recurrent Neural Network}
\newacronym{us}{US}{Ultra Sound imaging}
